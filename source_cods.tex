\defverbatim[colored]\newappGI{
  \begin{lstlisting}[language=bash,basicstyle=\ttfamily,
    keywordstyle=\color{red}]
   mkdir Rails_Group_1 
   cd Rails_Group_1
   rails new helldesk
   rails server
  \end{lstlisting}
}

\defverbatim[colored]\newappGII{
  \begin{lstlisting}[language=bash,basicstyle=\ttfamily,
    keywordstyle=\color{red}]
   mkdir Rails_Group_2 
   cd Rails_Group_2
   rails new helldesk
   rails server
  \end{lstlisting}
}


\defverbatim[colored]\newappworking{
  \begin{lstlisting}[language=bash,basicstyle=\ttfamily,
    keywordstyle=\color{red}]
   mkdir Rails_Group_2 
   cd Rails_Group_2
   rails new helldesk
   bundle install
   rails server
  \end{lstlisting}
}

\defverbatim[colored]\modelexample{
  \begin{lstlisting}[language=ruby,basicstyle=\ttfamily,
    keywordstyle=\color{red}]
    class User < ActiveRecord::Base
      validates :name, :presence => true,
                :uniqueness => true
      validates :password, :confirmation => true
      ...
    end
\end{lstlisting}
}

\defverbatim[colored]\viewexample{
  \begin{lstlisting}[language=ruby,basicstyle=\tiny,
    keywordstyle=\color{red}]
<% provide(:title, 'Index') %>
<h1>Listing Users</h1>
<% if notice %>
  <p id="notice"><%= notice %></p>
<% end %>
<table>
<tbody>
    <% @users.each do |user| %>
      <tr>
        <td><%= user.name %></td>
        <td><%= user.email %></td>
        <div class="actions">
            <td><%= link_to 'Show', user %></td>
            <td><%= link_to 'Edit', edit_user_path(user) %></td>
            <td><%= link_to 'Destroy', user, method: :delete, data: { confirm: 'Are you sure?' } %></td>
        </div>
      </tr>
    <% end %>
  </tbody>
</table>
<br>
<%= link_to 'New User', new_user_path %>
\end{lstlisting}
}

\defverbatim[colored]\classexample{
  \begin{lstlisting}[language=ruby,basicstyle=\tiny,
    keywordstyle=\color{red}]
    class TestClass
       def initialize
         @instance_var=10
       end

       def method
         put 'I am test class'
       end

       def instance_var
         return @instance_var
       end

       def instance_var=(var)
         @instance_var = var
       end
    end
\end{lstlisting}
}


\defverbatim[colored]\classexampleII{
  \begin{lstlisting}[language=ruby,basicstyle=\small,
    keywordstyle=\color{red}]
    class TestClass
       attr_accessor :instance_var
       def initialize
         @instance_var=10
       end

       def method
         put 'I am test class'
       end

    end
\end{lstlisting}
}

\defverbatim[colored]\classexampleprivateI{
  \begin{lstlisting}[language=ruby,basicstyle=\tiny,
    keywordstyle=\color{red}]
    class TestClass
       attr_accessor :instance_var
       def initialize
         @instance_var=10
         method
       end
       
       private:

       def method
         put 'I am test class'
       end

    end
\end{lstlisting}
}

\defverbatim[colored]\classexampleprivateII{
  \begin{lstlisting}[language=ruby,basicstyle=\tiny,
    keywordstyle=\color{red}]
    class TestClass
       attr_accessor :instance_var
       def initialize
         @instance_var=10
         method
       end

       def method1
         put 'I am test class'
       end
       
       def method2
         put 'I am test class'
       end

       private :method, :method2

    end
\end{lstlisting}
}

\defverbatim[colored]\classexampleprotected{
  \begin{lstlisting}[language=ruby,basicstyle=\tiny,
    keywordstyle=\color{red}]
    class TestClass
       attr_accessor :instance_var
       def initialize
         @instance_var=10
         method
       end

       protected:

       def method1
         put 'I am test class'
       end
       
       def method2
         put 'I am test class'
       end
    end
\end{lstlisting}
}

\defverbatim[colored]\railsgenerator{
  \begin{lstlisting}[language=bash,basicstyle=\tiny,
    keywordstyle=\color{red}]
    rails generate controller static_content start about help
\end{lstlisting}
}

\defverbatim[colored]\railschangeroute{
  \begin{lstlisting}[language=bash,basicstyle=\small,
    keywordstyle=\color{red}]
    root 'static_content#start'
\end{lstlisting}
}

\defverbatim[colored]\railstopmenu{
  \begin{lstlisting}[language=ruby,basicstyle=\tiny,
    keywordstyle=\color{red}]
    <div id="left_menu">
        <ul>
            <li> <%= link_to "Issues", "#" %> </li>
            <li> <%= link_to "Admin", "#" %> </li>
        </ul> 
    </div>

    <div>
        <ul>
            <li> <%= link_to "Login", "#" %></li>
            <li> <%= link_to "About", "#" %></li>
            <li> <%= link_to "Help", "#" %></li>
        </ul>
    </div>
\end{lstlisting}
}


\defverbatim[colored]\railspartial{
  \begin{lstlisting}[language=ruby,basicstyle=\small,
    keywordstyle=\color{red}]
    <%= redren 'layouts/top_menu" %>
\end{lstlisting}
}

\defverbatim[colored]\railsprovide{
  \begin{lstlisting}[language=ruby,basicstyle=\small,
    keywordstyle=\color{red}]
    <% provide(:title, 'Index') %> 
\end{lstlisting}
}

\defverbatim[colored]\railslayouthandletitle{
  \begin{lstlisting}[language=ruby,basicstyle=\small,
    keywordstyle=\color{red}]
    <% unless yield(:title).empty? %> 
        <title>Helldesk2 | <%=yield(:title)%> </title> 
    <%else %>
        <title>Helldesk2</title> 
    <% end %> 
\end{lstlisting}
}


\defverbatim[colored]\railshelperI{
  \begin{lstlisting}[language=ruby,basicstyle=\tiny,
    keywordstyle=\color{red}]
    def provide_title(subtitle='')
     hell = "Helldesk"
     if subtitle.empty?
       hell
     else
       hell + " | " + subtitle
    end
  end

\end{lstlisting}
}

\defverbatim[colored]\routeupdateI{
  \begin{lstlisting}[language=ruby,basicstyle=\tiny,
    keywordstyle=\color{red}]
    controller :static_content do
       get 'start' => :start
       get 'help' => :help
       get 'about' => :about
    end

\end{lstlisting}
}



\defverbatim[colored]\railstitlehelperinlayout{
  \begin{lstlisting}[language=ruby,basicstyle=\small,
    keywordstyle=\color{red}]
    <title><%=provide_title(yield(:title))%></title>

\end{lstlisting}
}

\defverbatim[colored]\generatorscaffold{
  \begin{lstlisting}[language=bash,basicstyle=\tiny,
    keywordstyle=\color{red}]
    rails generate scaffold User name:string email:string hash_password:string salt:string
\end{lstlisting}
}

\defverbatim[colored]\databasemigrationuser{
  \begin{lstlisting}[language=bash,basicstyle=\small,
    keywordstyle=\color{red}]
    rake db:migrate
\end{lstlisting}
}

\defverbatim[colored]\railsreallinksI{
  \begin{lstlisting}[language=html,basicstyle=\tiny,
    keywordstyle=\color{red}]
    <div id="right_menu">
        <ul>
            <li> <%= link_to "Login", "#" %></li>
            <li> <%= link_to "About", :about %></li>
            <li> <%= link_to "Help", :help %></li>
        </ul>
    </div> 

\end{lstlisting}
}


\defverbatim[colored]\migrationuser{
  \begin{lstlisting}[language=ruby,basicstyle=\tiny,
    keywordstyle=\color{red}]
    class CreateUsers < ActiveRecord::Migration
    def change
     create_table :users do |t|
       t.string :name
       t.string :email
       t.string :hash_password
       t.string :salt

       t.timestamps null: false
      end
     end
    end  

\end{lstlisting}
}


\defverbatim[colored]\railsconsolestart{
  \begin{lstlisting}[language=ruby,basicstyle=\small,
    keywordstyle=\color{red}]
    rails console development
\end{lstlisting}
}

\defverbatim[colored]\railsimportsha{
  \begin{lstlisting}[language=ruby,basicstyle=\small,
    keywordstyle=\color{red}]
    require 'digest/sha2' 
\end{lstlisting}
}

\defverbatim[colored]\railsconsoleencrypt{
  \begin{lstlisting}[language=ruby,basicstyle=\small,
    keywordstyle=\color{red}]
    Digest::SHA2.hexdigest(``example_word'')
\end{lstlisting}
}

\defverbatim[colored]\railsusermodelI{
  \begin{lstlisting}[language=ruby,basicstyle=\tiny,
    keywordstyle=\color{red}]
    class User < ActiveRecord::Base
      validates :name, :presence => true, :uniqueness => true

\end{lstlisting}
}

\defverbatim[colored]\railsusermodelII{
  \begin{lstlisting}[language=ruby,basicstyle=\tiny,
    keywordstyle=\color{red}]
    require 'digest/sha2'
    class User < ActiveRecord::Base
      validates :name, :presence => true, :uniqueness => true
      validates :password, :confirmation => true
      attr_accessor :password_confirmation
      attr_reader :password
      validate :password_must_be_present

      def User.encrypt_password(password, salt)
        Digest::SHA2.hexdigest(password + "slowo" + salt) 
      end

      def password=(password)
        @password = password
        if password.present?
          generate_salt
          self.hash_password = self.class.encrypt_password(password,self.salt)
        end
      end

      def User.authenticate(name,password)
        if user = User.find_by_name(name)
          if user.hash_password == encrypt_password(password, user.salt)
            user
          end
        end
      end

\end{lstlisting}
}

\defverbatim[colored]\railsusermodelIII{
  \begin{lstlisting}[language=ruby,basicstyle=\tiny,
    keywordstyle=\color{red}]
      private

      def password_must_be_present
        errors.add(:password, "Missing password") unless hash_password.present?
      end

      def generate_salt
        self.salt = self.object_id.to_s + rand.to_s
      end
    end
\end{lstlisting}
}


\defverbatim[colored]\railsusercontroller{
  \begin{lstlisting}[language=ruby,basicstyle=\tiny,
    keywordstyle=\color{red}]
    # Never trust parameters from the scary internet, only allow the white list through.
    def user_params
      params.require(:user).permit(:name, :email, :password, :password_confirmation)
    end 
\end{lstlisting}
}

\defverbatim[colored]\railsuserviewform{
  \begin{lstlisting}[language=ruby,basicstyle=\tiny,
    keywordstyle=\color{red}]
  <div class="field">
    <%= f.label :name %><br>
    <%= f.text_field :name %>
  </div>
  <div class="field">
    <%= f.label :email %><br>
    <%= f.text_field :email %>
  </div>
  <div class="field">
    <%= f.label :password %><br>
    <%= f.password_field :password %>
  </div>
  <div class="field">
    <%= f.label :password_confirmation %><br>
    <%= f.password_field :password_confirmation %>
  </div>
  <div class="actions">
    <%= f.submit %>
  </div>

\end{lstlisting}
}

